\documentclass[a4paper,12pt]{article}
\usepackage[utf8]{inputenc}
\usepackage[unicode]{hyperref}
\usepackage[russian]{babel}
\usepackage{geometry}
\geometry{margin=1in}

\title{Комплексный отчет по стратегии экспансии на фармацевтический рынок Турции}
\author{Rostislav Lokhov and Miroslav Slobodyanuk}
\date{}

\begin{document}

\maketitle

\section*{Резюме}
Этот отчет представляет собой тщательно разработанную стратегию выхода на растущий фармацевтический рынок Турции. Основное внимание уделено онкологическим и биофармацевтическим продуктам, которые способны обеспечить значительное улучшение качества жизни пациентов. В документе представлены расширенные аналитические данные, включая комплексные оценки на основе 3C, PESTEL и SWOT анализов, а также детализированные финансовые прогнозы, созданные на основе данных из \texttt{main.py}. Основные рекомендации включают создание совместных предприятий, локализацию производства и долгосрочное построение конкурентных преимуществ.

Источники данных \textbf{[source]} включают \texttt{ALL\_DATA.txt}, \texttt{6\_months\_report.txt}, \texttt{12+\_report.txt}, \texttt{FILTERED.txt} и \texttt{main.py}.

\section{Раздел 1: Стратегический анализ}

\subsection{Целевой регион и продукты}
\begin{itemize}
    \item \textbf{Регион}: Турция. По прогнозам, объем фармацевтического рынка достигнет \textbf{2.09~billion~USD} в 2024 году с увеличением до \textbf{12.23~billion~USD} к 2032 году (CAGR \textbf{4.50\%}). Источник: \texttt{ALL\_DATA.txt}.
    \item \textbf{Продукты}: Основное внимание уделено онкологическим препаратам и биофармацевтическим решениям, включая моноклональные антитела. Источник: \texttt{FILTERED.txt}.
\end{itemize}

\subsection{Оценка рынка}
\subsubsection{Методология 3C}
\begin{itemize}
    \item \textbf{Компания}: Высокая компетенция в области биофармацевтических инноваций, подтвержденная успешной работой на других рынках. Источник: \texttt{ALL\_DATA.txt}.
    \item \textbf{Клиенты}: Увеличение спроса на онкологические препараты. Ожидается, что к 2040 году зарегистрировано \textbf{392,000 новых случаев рака}. Источник: \texttt{FILTERED.txt}.
    \item \textbf{Конкуренты}: 25 ведущих компаний контролируют 80\% рынка. Источник: \texttt{FILTERED.txt}.
\end{itemize}

\subsubsection{PESTEL-анализ}
\begin{itemize}
    \item \textbf{Политика}: Государственная поддержка локального производства через инициативы и субсидии. Источник: \texttt{FILTERED.txt}.
    \item \textbf{Экономика}: Высокая инфляция и волатильность валюты требуют стратегического подхода.
    \item \textbf{Социальные факторы}: Старение населения и рост числа пациентов с хроническими заболеваниями. Источник: \texttt{12+\_report.txt}.
    \item \textbf{Технологии}: Значительное внимание уделяется государственным инвестициям в исследования и разработки. Источник: \texttt{12+\_report.txt}.
\end{itemize}

\subsubsection{SWOT-анализ}
\begin{itemize}
    \item \textbf{Сильные стороны}: Высокотехнологичные продукты, возможности для локализации.
    \item \textbf{Слабые стороны}: Регуляторные барьеры и высокая конкуренция.
    \item \textbf{Возможности}: Рост сегмента биосимиляров, налоговые стимулы.
    \item \textbf{Угрозы}: Волатильность экономики и изменение регуляторной политики.
\end{itemize}

\subsection{Рекомендации по выходу на рынок}
Создание совместного предприятия с локальными компаниями, такими как \texttt{AbdiBio}, для быстрого выхода на рынок и минимизации рисков. Источник: \texttt{FILTERED.txt}.

\section{Раздел 2: Продуктовая стратегия}

\subsection{Ценностное предложение}
\begin{itemize}
    \item \textbf{Ключевые преимущества}: Препараты, улучшающие выживаемость пациентов с онкологическими заболеваниями на \textbf{30\%}. Источник: \texttt{FILTERED.txt}.
    \item \textbf{УТП}: Инновационные биофармацевтические продукты с минимальными побочными эффектами и высокой эффективностью.
\end{itemize}

\subsection{Карта пути пациента}
\begin{itemize}
    \item \textbf{Основные этапы}:
    \begin{enumerate}
        \item Осознание симптомов.
        \item Обращение к врачу и диагностика.
        \item Назначение лечения.
        \item Покупка препаратов.
        \item Лечение и мониторинг.
        \item Корректировка терапии.
    \end{enumerate}
    \item \textbf{Оптимизации}: Телемедицинские платформы для повышения доступности лечения. Источник: \texttt{FILTERED.txt}.
\end{itemize}

\subsection{Ценообразование}
\begin{itemize}
    \item \textbf{Методология}: Ценообразование на основе ценности (Value-Based Pricing), адаптированное под локальные экономические условия.
    \item \textbf{Диапазон цен}: Соответствует уровню платежеспособности и компенсационным механизмам. Источник: \texttt{12+\_report.txt}.
\end{itemize}

\section{Раздел 3: Операционная стратегия}

\subsection{Локализация и производство}
\begin{itemize}
    \item \textbf{Стратегия}: Создание СП с минимальными затратами \textbf{2~million~USD} и дополнительно \textbf{1~million~USD} на каждые 10,000 единиц. Источник: \texttt{main.py}.
    \item \textbf{Партнеры}: Сотрудничество с исследовательскими центрами и университетами Турции. Источник: \texttt{FILTERED.txt}.
\end{itemize}

\subsection{Структура затрат}
\begin{itemize}
    \item \textbf{Капитальные вложения}: \textbf{70~million~TRY}.
    \item \textbf{Переменные издержки}: \textbf{20~USD} на единицу. Источник: \texttt{main.py}.
\end{itemize}

\section{Заключение}
Настоящая стратегия предоставляет всеобъемлющую основу для успешного выхода на фармацевтический рынок Турции. Локализация, стратегические альянсы и инновационные подходы позволят компании добиться устойчивого роста и конкурентных преимуществ.

\end{document}
